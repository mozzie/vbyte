\chapter{Introduction}
Enormous datasets are a common case in today's applications. Compressing the datasets is beneficial, because they 
naturally decrease memory requirements but also are faster when compressed data is read from disk \citep{Zob95}.

One of the leading methods of data compression is variable-length coding \citep{Sal99}, where frequent sequences of data
are represented with shorter codewords. Because the sequences of data have different lenghts when compressed, it is 
not trivial to determine the exact location of a certain element. If this is required, the usual data compression algorithms are
inefficient. Fortunately this is not a requirement compression algorithms usually need to fulfill. 

However, random access of compressed data is needed in compressed data structures. In most compression methods, the only way 
to this is to decompress data from the beginning. An integer compressing method with fast random access is explained and compared
existing state-of-the-art methods.

\chapter{Variable-byte encoding}

Variable-byte encoding \citep{Wil99} (VB) is a method of compressing integers via omitting leading zero bits. A good data set for 
VB encoding is a list of different sized numbers. As an example, an inverted index used by document search engine stores word frequencies and 
locations in documents. VB saves space here because index numbers need to support large values, but are usually significantly smaller. 
It is also proven to improve search speed, because less bytes need to be read from a hard drive \citep{Sch02}.

VB splits an integer into blocks of $b$ bits. A continuation bit is added to the block to form a chunk $c$ and then chunk non-empty chunks are stored.
The continuation bit is set to 1 for the last block of the integer. During decoding, chunks are read and their blocks are joined until a continuation 
bit is set to 1. An example implementation is showcased in Figure~\ref{fig:vbyte}.

\begin{figure}[h!] 
\centering 
blblblblb
testataan
tätä
\caption{VB encoding and decoding.\label{fig:vbyte}}
\end{figure}

Small lengths of $c$ can yield better
compression rate at the cost of more bit manipulation, while longer chunks need less bit manipulation and offer less effective compression. 
Generally chunk length of 8 has been used because it gives a good average and handling bytes is convenient \citep{IRbook}.

bl 
It splits an integer into blocks of $b$ bits and then encodes
the block as $b + c$ where $c$ is a bit denoting whether $b$ has the most significant bits of the number or not. 

This method loses in compression performance to other methods \citep{Bri09}, but decoding is fast.



\chapter{Directly addressable codes?}
Rank and select are two functions that work on bit arrays. Rank(i) gives the sum of 1 bits from the beginning of the bit array and select(i) gives
the index of ith 1 bit in the bit array. Both functions work in constant time (citation?) and they require only a few percents of extra space over
the data. The extra bits $c$ added by variable-byte encoding conveniently create a bit array, where 1's represent the endings of numbers. An effective
version of random access has already been introduced \citep{Bri09}. 

Random access with select query is also possible.  By separating the $c$ bit array and $b$ block array, $b$ contains variable-byte integers 
in readable form. Another upside is that functions next(i) and previous(i) are conveniently available. Rank implementation has 

- explain how random access is good 

\chapter{Previous Work}
bl

\chapter{Algorithm}

 - modifications to basic implementation
 
\chapter{Results}
 - comparison to basic implementation + Bri09
\begin{table}
\centering
\caption{Results with 100k entries (in milliseconds).\label{table:results}}
\begin{tabular}{l||l c c c c r} 
Experiment & 128 & 256 & 32768 & 65536 & $2^{24}$ & $2^{32} -1$\\ 
\hline \hline 
$7bit VByte encoding$ & 34.97 & 49 & 53.04 & 52.18 & 53.08 & 76.21\\
$8bit VByte encoding$ & 33.57 & 32.47 & 42.96 & 43.11 & 46.15 & 65.14\\
$7bit VByte encoding with array$ & 33.39 & 46.85 & 51.24 & 49.03 & 48.93 & 66.84 \\
$8bit VByte encoding with array$ & 32.52 & 31.88 & 41.54 & 39.94 & 41.15 & 52.86 \\

\hline
%
\end{tabular}
\end{table}

\begin{table}
\centering
\caption{Results with 1M entries (in milliseconds).\label{table:results}}
\begin{tabular}{l||l c c c c r} 
Experiment & 128 & 256 & 32768 & 65536 & $2^{24}$ & $2^{32} -1$ \\ 
\hline \hline 
$7bit VByte encoding$ & 38.17 & 55.09 & 64.38 & 65.36 & 68.08 & 159 \\
$8bit VByte encoding$ & 37.09 & 37.75 & 53.44 & 54.6 & 59.32 & 148.7\\
$7bit VByte encoding with array$ & 38.09 & 55.42 & 62.22 & 61.25 & 71.72 & 135.01\\
$8bit VByte encoding with array$ & 36.13 & 36.83 & 50.58 & 50.73 & 56.93 & 103.18\\

\hline
%
\end{tabular}
\end{table}

\chapter{Conclusion}
 - here

\chapter{Future work}
 - something to improve / research?


\section{Figures}
Figure~\ref{fig:logo} gives an example how to add figures to the document. Remember always to cite the figure in the main text.

\begin{figure}[h!] 
\centering 
\includegraphics[width=0.3\textwidth]{HY-logo-ml.png}
\caption{University of Helsinki flame-logo for Faculty of Science.\label{fig:logo}}
\end{figure}

\section{Tables}

Table~\ref{table:results} gives an example how to report experimental results. Remember always to cite the table in the main text. 


\chapter{Citations}

\section{Citations to literature}

References are listed in a separate .bib-file. In this case it is named \texttt{bibliography.bib} including the following content:
\begin{verbatim}

@article{einstein,
    author =       "Albert Einstein",
    title =        "{Zur Elektrodynamik bewegter K{\"o}rper}. ({German})
        [{On} the electrodynamics of moving bodies]",
    journal =      "Annalen der Physik",
    volume =       "322",
    number =       "10",
    pages =        "891--921",
    year =         "1905",
    DOI =          "http://dx.doi.org/10.1002/andp.19053221004"
}
 
@book{latexcompanion,
    author    = "Michel Goossens and Frank Mittelbach and Alexander Samarin",
    title     = "The \LaTeX\ Companion",
    year      = "1993",
    publisher = "Addison-Wesley",
    address   = "Reading, Massachusetts"
}
 
@misc{knuthwebsite,
    author    = "Donald Knuth",
    title     = "Knuth: Computers and Typesetting",
    url       = "http://www-cs-faculty.stanford.edu/%7Eknuth/abcde.html"
}
\end{verbatim}

In the last reference url field the code \verb+%7E+ will translate into \verb+~+ once clicked in the final pdf.

References are created using command \texttt{\textbackslash cite\{einstein\}}, showing as \citep{einstein}. Other examples: \citep{latexcompanion,knuthwebsite}.

Citation style can be negotiated with the supervisor. See some options in \url{https://www.sharelatex.com/learn/Bibtex_bibliography_styles}.

\section{Crossreferences}

Appendix~\ref{appendix:model} on page~\pageref{appendix:model} contains some additional material.

\chapter{From tex to pdf}

In Linux, run \texttt{pdflatex filename.tex} and \texttt{bibtex filename.tex} repeatedly until no more warnings are shown. This process can be automised using make-command.
 
\chapter{Conclusions\label{chapter:conclusions}}

It is good to conclude with a summary of findings. You can also use separate chapter for discussion and future work. These details you can negotiate with your supervisor.

