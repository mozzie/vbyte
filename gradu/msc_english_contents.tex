\chapter{Introduction}

The following gives some superficial instructions for using this template for a Master's thesis. For guidelines on thesis writing you can consult various sources, for example, the Bachelor thesis template.

The thesis should have an introduction chapter. Other chapters can be named according to the topic. In the end, some summary chapter is needed; see Chapter~\ref{chapter:conclusions} for an example.

\chapter{Figures and Tables}

\section{Figures}
Figure~\ref{fig:logo} gives an example how to add figures to the document. Remember always to cite the figure in the main text.

\begin{figure}[h!] 
\centering 
\includegraphics[width=0.3\textwidth]{HY-logo-ml.png}
\caption{University of Helsinki flame-logo for Faculty of Science.\label{fig:logo}}
\end{figure}

\section{Tables}

Table~\ref{table:results} gives an example how to report experimental results. Remember always to cite the table in the main text. 

\begin{table}
\centering
\caption{Results with 100k entries (in milliseconds).\label{table:results}}
\begin{tabular}{l||l c c c c r} 
Experiment & 128 & 256 & 32768 & 65536 & $2^{24}$ & $2^{32} -1$\\ 
\hline \hline 
$7bit VByte encoding$ & 34.97 & 49 & 53.04 & 52.18 & 53.08 & 76.21\\
$8bit VByte encoding$ & 33.57 & 32.47 & 42.96 & 43.11 & 46.15 & 65.14\\
$7bit VByte encoding with array$ & 33.39 & 46.85 & 51.24 & 49.03 & 48.93 & 66.84 \\
$8bit VByte encoding with array$ & 32.52 & 31.88 & 41.54 & 39.94 & 41.15 & 52.86 \\

\hline
%
\end{tabular}
\end{table}

\begin{table}
\centering
\caption{Results with 1M entries (in milliseconds).\label{table:results}}
\begin{tabular}{l||l c c c c r} 
Experiment & 128 & 256 & 32768 & 65536 & $2^{24}$ & $2^{32} -1$ \\ 
\hline \hline 
$7bit VByte encoding$ & 38.17 & 55.09 & 64.38 & 65.36 & 68.08 & 159 \\
$8bit VByte encoding$ & 37.09 & 37.75 & 53.44 & 54.6 & 59.32 & 148.7\\
$7bit VByte encoding with array$ & 38.09 & 55.42 & 62.22 & 61.25 & 71.72 & 135.01\\
$8bit VByte encoding with array$ & 36.13 & 36.83 & 50.58 & 50.73 & 56.93 & 103.18\\

\hline
%
\end{tabular}
\end{table}

\chapter{Citations}

\section{Citations to literature}

References are listed in a separate .bib-file. In this case it is named \texttt{bibliography.bib} including the following content:
\begin{verbatim}
@article{einstein,
    author =       "Albert Einstein",
    title =        "{Zur Elektrodynamik bewegter K{\"o}rper}. ({German})
        [{On} the electrodynamics of moving bodies]",
    journal =      "Annalen der Physik",
    volume =       "322",
    number =       "10",
    pages =        "891--921",
    year =         "1905",
    DOI =          "http://dx.doi.org/10.1002/andp.19053221004"
}
 
@book{latexcompanion,
    author    = "Michel Goossens and Frank Mittelbach and Alexander Samarin",
    title     = "The \LaTeX\ Companion",
    year      = "1993",
    publisher = "Addison-Wesley",
    address   = "Reading, Massachusetts"
}
 
@misc{knuthwebsite,
    author    = "Donald Knuth",
    title     = "Knuth: Computers and Typesetting",
    url       = "http://www-cs-faculty.stanford.edu/%7Eknuth/abcde.html"
}
\end{verbatim}

In the last reference url field the code \verb+%7E+ will translate into \verb+~+ once clicked in the final pdf.

References are created using command \texttt{\textbackslash cite\{einstein\}}, showing as \citep{einstein}. Other examples: \citep{latexcompanion,knuthwebsite}.

Citation style can be negotiated with the supervisor. See some options in \url{https://www.sharelatex.com/learn/Bibtex_bibliography_styles}.

\section{Crossreferences}

Appendix~\ref{appendix:model} on page~\pageref{appendix:model} contains some additional material.

\chapter{From tex to pdf}

In Linux, run \texttt{pdflatex filename.tex} and \texttt{bibtex filename.tex} repeatedly until no more warnings are shown. This process can be automised using make-command.
 
\chapter{Conclusions\label{chapter:conclusions}}

It is good to conclude with a summary of findings. You can also use separate chapter for discussion and future work. These details you can negotiate with your supervisor.
