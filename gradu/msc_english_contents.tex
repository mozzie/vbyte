\chapter{Introduction}
Enormous datasets are a common case in today's applications. Compressing the datasets is beneficial, because they 
naturally decrease memory requirements but also are faster when compressed data is read from disk \citep{Zob95}. 

One of the leading methods of data compression is variable-length coding \citep{Sal99}, where frequent sequences of data
are represented with shorter codewords. Because the sequences of data have different lengths when compressed, it is 
not trivial to determine the exact location of a certain element in the compressed data. If this is required, the usual 
data compression algorithms are inefficient. Fortunately this is not a requirement compression algorithms usually need to fulfill. However, 
random access of compressed data is useful in compressed data structures. It saves storage space, bandwidth and increases the likelihood
of data already being in cache \citep{Sch02}. In most compression methods, the only way to do this is to decompress data from the beginning. 

A variable-byte encoding based integer compression method with constant time random access was first introduced by \citep{Bri09}. They used clever block 
reorganizing and $rank$ data structure to achieve random access. Their solution is currently the only published solution for the problem and it 
has been widely adopted. 

An alternative solution with $select$ data structure is proposed and explained in detail later in this thesis. Comparison to $rank$ implementation by Brisaboa et al is 
shown with different implementations of $rank$ and $select$ and several different data sets. Proposed method does not use data 
reorganizing, but rather capitalizes on the assumption that data is encoded the same way it is in the source. Because variable-byte encoding does not assign 
new codewords to data when encoding, data can be read straight from the memory and reassembling the integer from blocks is not needed.

$Rank$ method is superior when dataset contains mostly small integers. With larger integers, multiple $rank$ calls are needed and $select$ method performs better. Additionally, the
proposed $select$ method offers quick access to next or previous element.




\chapter{Variable-bytes}

Variable-byte (VB) encoding \citep{Wil99} is a method for compressing unsigned integers via omitting leading zero bits that would be present in a longer fixed 
width word. In normal data sets, VB encoding loses in compression performance to generic algorithms like Huffman encoding or Lempel-Ziv encoding, but 
is generally faster to decode \citep{Bri09} and sometimes preferred due to its simplicity. 

A good data set for VB encoding is a list of mostly small integers with a need to support larger ones. A search engine may use an inverted index of 
words in documents. For each word, a list of document IDs where the word appears is stored. It may also store locations of the word in document for 
advanced search purposes. Usually these lists are preprocessed and stored as an inverted index or gaps, storing the difference to previous number 
instead of the actual number \citep{Man08}. Common words have a lot of entries in these lists, but because of gap storing the integers are small. 
In contrast, rare words have only a few entries but the integers stored are larger. These lists are excellent data sets for VB encoding. 

Variable-byte encoding originates from and is used in MIDI music file format \citep{Mid96} and several applications have a similar implementation of VB. Apache 
Lucene has vInt datatype. Wireless Application Protocol has a variable length unsigned integer uintvar, Google Protocol Buffers has a Base 128 Varint \citep{GooPB},
 Microsoft .Net framework offers "7BitEncodedInt" in BinaryReader and BinaryWriter classes and IBM DB2 has a variable byte  \citep{Bha09}.

Later, VB encoding was to 
be found efficient in compressing integer lists. It was first experimented as a tool for compressing inverted index lists of word locations in documents \citep{Sch02}. 
That yielded excellent records, and since then many different approaches have been introduced. 

More recent studies have taken a look into the machine code level for VB and applied SIMD (Single instruction, multiple data) instructions to VB \citep{Lem18,Pla15}. The bit 
operations in VB are simple and therefore modifying the code to use SIMD instructions is straightforward and the speed improvements are significant. 

Elias Delta and Gamma codes \citep{Eli75} are popular encoding methods for forementioned data sets. Their encoding process assigns short bit array codes to small integers, 
which is why they outperform VB encoding on datasets with a lot of small integers \citep{Wil99}, such as inverted indexes \citep{Anh05, Pib19}. In this thesis, the compressable data sets will have
the focus on slightly larger, unordered numbers.

\section{VB encoding}
VB encoding splits each integer into blocks of $b$ bits and adds a continuation bit to the block to form chunks of length $1+b$. The continuation bit is set to 1 only
on the block containing the least significant bits of the integer. This information is used in decoding to signal if the current integer continues in the next chunk. 

For example, the standard bit representation of an unsigned integer 42 is \texttt{00000000 00101010}. When split to blocks of length 4, it becomes \texttt{0000 0000 0010 1010}. The empty blocks 
are removed and then continuation bits are added. 1 is added to the least significant block and 0 to the other block, resulting in \texttt{00010 11010}. A pseudo code with block length 4 is shown in Figure~\ref{vbyte_enc}. Prepend adds an element to the beginning of the list 
and extend adds all the elements of the second list to the end of the first list.

\begin{figure}[ht]
\centering
  \begin{minipage}{0.5\linewidth}
\begin{algorithmic}[H]
\Function{VBEncodeNumber}{$n$}
\State $bytes\gets $ list
\While{true}
\State \Call{prepend}{bytes,$n \bmod 128$}
\If{$n < 128$} \State break \EndIf
\State $n\gets n$ div $128$
\EndWhile
\State $bytes$[\Call{len}{$bytes$}-1] += $128$
\State \textbf{return} bytes
\EndFunction
\medskip
\medskip
\Function{VBEncode}{$numbers$}
\State $bytestream\gets $ list
\ForEach {$n \in numbers$}
\State $bytes \gets$ \Call{VBEncodeNumber}{$n$}
\State \Call{extend}{$bytestream$,$bytes$}
\EndFor
\State \textbf{return} bytestream
\EndFunction

\end{algorithmic}
\end{minipage}
\captionof{figure}{VByte encoding} \label{vbyte_enc}
\end{figure}

Smaller block length can yield a better compression rate at the cost of more bit manipulation, while bigger block lenghts need less bit manipulation and 
offer less effective compression. On the other hand, bigger block length means a smaller percentage of added continuation bits.
Generally block length of 7 has been used because it tends to perform well on average and handling chunks as bytes is 
convenient \citep{Man08}.


\begin{table}
\centering
\begin{tabular}{l||c c c c c} 
Original number & first block & second block & third block & fourth block &\\ 
\hline \hline 
4  & \texttt{\underline{1}0100}    &                             &                           &  &  \\
17  & \texttt{\underline{0}0001}   & \texttt{\underline{1}0001}  &                           &  &  \\
620  & \texttt{\underline{0}1100}  & \texttt{\underline{0}0110} & \texttt{\underline{1}0010} &  &  \\
60201 & \texttt{\underline{1}1001} & \texttt{\underline{0}0010} & \texttt{\underline{0}1011} & \texttt{\underline{1}1110} &  \\

\hline
\end{tabular}
\caption{VByte encoded integers, block size 4. Continuation bit underlined.\label{table:vbytes}}
\end{table}

\section{VB decoding}
VB decoding reverses the encoding steps: encoded chunks are read until a chunk with 1 as continuation bit is found. Continuation bits 
are removed from all the chunks and the blocks are concatenated to form the original integer. A pseudo code implementation of VB decoding with block length of 7 is shown in Figure~\ref{vbyte_dec}.
Append adds an element to the end of the list. If block length is changed, additional bit operation steps when reading the data may be needed.

As how the encoding example ended, the encoded message was \texttt{00010 11010} with block length of 4 and the goal is to decode a 16 bit unsigned integer. The block from the first chunk is 
extracted and added to $n$, making $n = $\texttt{10} (bit representation of 2).
The continuation bit was 0 in this chunk, which means the encoded integer continues to the next block. A bitwise shift to the left equal to block length is applied to $n$, 
changing $n = $\texttt{100000} (bit representation of 32). Then the chunk reading process is repeated. The block of the next chunk is added to $n$, making 
$n = $\texttt{101010} (bit representation of 42). The continuation bit of this chunk is 1, which means the decoding for this number is complete. 

\begin{figure}[ht]
\centering
  \begin{minipage}{0.5\linewidth}
\begin{algorithmic}[H]
\Function{VBDecode}{$bytestream$}
\State $numbers\gets $ list
\State $n\gets 0$
\ForEach {$b \in bytestream$}
\If{$b < 128$}
\State $n\gets 128\times n $ + $b$
\Else
\State $n\gets 128\times n $ + $b$ - $128$
\State \Call{append}{numbers,n}
\State $n\gets 0$
\EndIf
\EndFor
\State \textbf{return} numbers
\EndFunction
\end{algorithmic}
\end{minipage}
\caption{VByte decoding} \label{vbyte_dec}
\end{figure}



\chapter{Directly addressable codes} \label{chapter:DAC}

The ability to handle large amounts of data fast is one of the key challenges in the field of search engines. Compressed data structures are applied to fit the 
data into cache, memory or even hard drive. Direct access to any element in a compressed list or array is one of the usual requirements in compressed data 
structures. It is not natively possible to decode the i-th element in variable-byte compression algorithms, because the position of the element in the compressed
list depends on the length of the preceding compressed data. Direct access is achievable with supporting data structures. A naive solution is to store the location 
of each element to an array, but it adds a very large overhead which removes the benefit of compression.

\section{Rank and Select}
Rank and select are two succinct data structure operations which operate on a bit array B. $Rank_1$(B, i) gives the sum of 1 bits between B[0] and B[i] and 
$select_1$(B, i) gives the index of i-th 1 bit in B. Both operations can be implemented to work in constant time \citep{gbmp2014sea}. To use rank or select in indexing, the
set 1 bits in B should reflect to the element locations in the encoded data. 

For most compression algorithms, this requires B to be created in addition to the existing data and the length of B usually 
has to be close to the length of the data, because encoded element lenghts vary. VB encoding has several advantages with search and rank: the data is compressed in 
blocks of equal length which significantly shrinks the length of B. Moreover the bit array C formed from the continuation bits already stores the locations of items and works
as the needed indexing array. In this case, $rank_1$(C, i) would give the sum of end blocks up to i-th index and $select_1$(C, i) would give the location of the ending block 
of i-th compressed element.

\section{Rank and Select implementation}

The rank and select implementations used in this thesis are from C++ library 'SDSL-lite' by \citep{gbmp2014sea}. The library has an implementation of a bit array and  several implementations
 of both rank and select to support the bit array. Also a few useful functions were used during the implementation phase. Table~\ref{table:supportsize} has $rank$ and $select$ 
size requirements of implementations used in this experiment. Both rank implementations have a constant space requirement over the bit array, while select's needed size depends on 
the number of 1's in the data.

The data structure of rank has two layers. First layer is the superblock array. For every 512th bit, the number of 1's from the beginning of the array is stored to the superblock array.
The second layer is the relative count block. It stores a relative count of 1's for every 64th bit. To calculate rank(i), the superblock index is s = i/512 and 
relative count block index is r = i-s / 64, both being integer divisions. Then number of 1 bits from index r to i are calculated. All three values added together add up to rank(i). 
Rankv5 is a lighter structure: it's superblock size is 2048 and relative counts are taken for every 384th bit. This causes the final bit calculation to be more costly, but reduces 
memory requirement to a quarter.

$Select$ data structure works similarly to rank. The index location of every 4096th set bit is stored in the superblock and location of every 64th set bit is stored 
relative to the superblock. With similar calculations, the location of closest 64th set bit is calculated and then bit array is iterated until required amount of bits is reached. 

In both cases, one function call always gets a value from the superblock and then from the superblock's relative block. The only differiating factor is the manual bit count from relative
block's index to wanted index. This is at most the size of one relative block and thus both of the functions work in constant time \citep{Gon05}.

\begin{table}
\centering
\caption{Memory requirements of SDSL rank and select data structures\label{table:supportsize}}
\begin{tabular}{l||c} 
Structure & Extra size taken\\ 
\hline \hline 
Rank   & \text{25\% of bit array} \\
Rankv5 & \text{6.25\% of bit array}\\
Select & \text{~8-12\% of bit array}\\
\hline
\end{tabular}
\end{table}



\section{DAC via rank}
Directly addressable codes in VB was first introduced by \citep{Bri09}. In their solution, the chunks are sorted in separate arrays by their significance. For each integer, the block 
of the least significant chunk is stored in the first array $A_1$, and its bit to the first bit array $B_1$. Then if the number was stored in multiple chunks, the block of the next chunk is
stored to $A_2$ and the bit to $B_2$ and so on. After the data is set, rank data structure is built for each bit array. The data structure is implicit and does not require additional space apart
from the $rank$ structure. Figure~\ref{bris_ds} contains a visualization of the data structure.



\begin{figure}[ht]
\centering
\begin{tikzpicture}
[node distance=0pt,
 start chain = A going right,
 arrow/.style = {draw=#1,-{Stealth[]}, 
                shorten >=1mm, shorten <=1mm}, % styles of arrows
 arrow/.default = black,
 X/.style = {rectangle, draw,% styles of nodes in string (chain)
                minimum width=7ex, minimum height=4ex,
                outer sep=0pt, on chain}
 ]
\node[X, join] (start) {$C_{1,2}C_{1,1}$};
\node [above=1ex of start]{The chunk array. $C_{i,j}$ = $B_{i,j}$ : $A_{i,j}$};
\node [left=1ex of start]{\textbf{C =}};
\node[X] {$C_{2,1}$};
\node[X] {$C_{3,3}C_{3,2}C_{3,1}$};
\node[X] {$C_{4,2}C_{4,1}$};
\node[X] {$C_{5,1}$...};

\node[X, below=5ex of start] (a1) {$A_{1,1}$};
\node [left=1ex of a1]{\textbf{$A_1$ =}};
\node[X] {$A_{2,1}$};
\node[X] {$A_{3,1}$};
\node[X] {$A_{4,1}$};
\node[X] {$A_{5,1}$};
\node[X] {...};

\node[X, below=0ex of a1] (b1) {0};
\node [left=1ex of b1]{\textbf{$B_1$ =}};
\node[X] {1};
\node[X] {0};
\node[X] {0};
\node[X] {1};
\node[X] {...};
    

\node[X, below=5ex of b1] (a2) {$A_{1,2}$};
\node [left=1ex of a2]{\textbf{$A_1$ =}};
\node[X] {$A_{3,2}$};
\node[X] {$A_{4,2}$};
\node[X] {...};

\node[X, below=0ex of a2] (b2) {1};
\node [left=1ex of b2]{\textbf{$B_2$ =}};
\node[X] {0};
\node[X] {1};
\node[X] {...};

\node[X, below=5ex of b2] (a3) {$A_{3,3}$};
\node [left=1ex of a3]{\textbf{$A_3$ =}};
\node[X] {...};

\node[X, below=0ex of a3] (b3) {1};
\node [left=1ex of b3]{\textbf{$B_3$ =}};
\node[X] {...};


\end{tikzpicture}
\caption{Data structure by Brisaboa et al., visualized}\label{bris_ds}

\end{figure}

To decode an element from index i, first block is fetched from $A_1$[i]. Because each encoded element is composed of at least one block, the first block is obtainable via straight indexing. With small integers, this 
is all that is required. If the data has a lot of small integers, the $rank$ method has a huge advantage. In addition, the block length of this method can change within different array levels. This opens 
a possibility for further optimization.

However if the element was stored in multiple blocks, the index of the next block in $A_2$ is obtained from $i \gets rank_0$($A_1$, i). $Rank_0$ 
returns the number of zeros in the bit array preceding index i. In other words, the number of elements before $i$ that continue to the next array. 
This in turn means that $i$ now has the index of the element's next block in array $A_2$.
This process is repeated until the i-th bit of the bit array of current level is set. The resulting integer is constructed from the blocks of the fetched chunks. A pseudo code example of DAC with $rank$ is shown in Figure~\ref{bris_pseudo} 
with block length of 8.


\begin{figure}[ht]
\centering
\begin{minipage}{0.5\linewidth}
\begin{algorithmic}
\State $index \gets $ wanted index
\State $A \gets $ block arrays
\State $B \gets $ continuation bit arrays
\State $level \gets 0$
\State $number \gets 0$
\While {$B[level][index] = 0$}
\State $block \gets A[level][index]$
\State $number \gets number \mathbin{\ll} 8$
\State $number \gets number + block$
\State $index \gets \Call{Rank}{B[level], index}$
\State $level \gets level + 1$
\EndWhile
\State $block \gets A[level][index]$
\State $number \gets number \mathbin{\ll} 8$
\State $number \gets number + block$
\end{algorithmic}
\end{minipage}
\caption{Example pseudo code of DAC with rank by Brisaboa et al.} \label{bris_pseudo}

\end{figure}


\chapter{DAC with select query}

Using $select_1$ on the continuation bit array to achieve direct access is more intuitive than using $rank_1$. The element locations are already marked with 1's 
in $B$ and a single $select_1$ query gets the desired starting point, while the forementioned version \citep{Bri09} used one $rank$ query for each chunk beyond 
the first. Minimizing the amount of $select$ and $rank$ queries is important. They run in constant time but their impact is huge, because rest of the VB decoding 
consists of a few bit operations. 

To use $select_1$ with VB, continuation bits need to be separated from chunks to their own bit array and a $select_1$ structure built over it. Because every compressed element has 1
only on it's last block, $Select_1$(i) returns the location of the end byte of i-th element. Therefore the start of j-th element in the block array is at 
block $b_s$ = $select_1$(j-1) + 1. Implementation was simplified by using only block sizes 4 and 8 to prevent block splitting between bytes.

Unlike the standard VB encoding, the continuation bits are removed from the chunks and stored in their own array, which leaves the blocks in their own array. This allows the 
compressed number to be read from the memory block and removes the need of block concatenation. The data in blocks is written into memory as they appear in the original integer, so that 
when reading a word from the block byte array, the bits and bytes are already in correct order. 

For example, let the integer bit length be 32 and the original data be compressed to 5 blocks of 4 bits. First, the byte location of the first block is needed. $Select_1$(i) gets the index location
of needed block. Because the block length differs from byte length, the the two values need to be calculated. First, the starting byte location $s$ needs to be calculated 
$(s = select_1(i) \times 4) \div 8$, the latter operand being the integer division. Second, the first block might not be at the start of the byte $s$. This is why offset $o = (select_1(i) \times 4)$ $mod$ $8$
is needed (in the example case, o equals to either 0 or 4). Then a 32-bit word $w$ is read from memory from byte location $s$. 

At this point, $w$ contains the wanted bits, but also has extra bits from the previous and next compressed integers. If offset was not zero, there are trailing $o$ bits in $w$ from the previous integer. 
These can be conveniently removed by bit-shifting $w$ right for $o$ bits. Then a pre-calculated bitmask (0xFFFFF in this case) is applied and $w$ contains the requested value. Because most systems are
able to read memory only from byte addresses, there is a corner case where the value is stored in too many bytes for a word. In this setup, this happens if compressed length was 8 blocks and offset was 4.
Then another step of reading another byte and storing the remaining bits to $w$ is required. This is entirely avoidable if integer length is constrained. In this case maximum integer length would be 28 bits.

The most intuitive way to calculate block length of i-th element is from $select_1$(i+1) and subtract the previously calculated start block index. This however causes a second 
$select_1$ query, which is costly. A much faster option is to iterate forward in bit array until the next 1. Alternatively, the block length can be calculated by reading an integer 
from the bit array, aligning it's offset and counting the trailing zeros. Figure~\ref{select_pseudo} contains an example of VB decoding with DAC with $select$ and block size 8. 
Different block sizes need extra calculation to get the block location from the byte array. In the example, $CalculateLength$ returns the length of the number in blocks and $ByteMask(k)$ returns 
a bit mask for k bytes. 

\begin{figure}[ht]
\centering
\begin{algorithmic}
\State $i \gets $ wanted index
\State $A \gets $ block byte array
\State $B \gets $ continuation bit array
\State $begin \gets \Call{Select}{index}$
\State $len \gets \Call{CalculateLength}{begin}$
\State $mask \gets \Call{ByteMask}{len}$
\State $word \gets A[begin]$ \Comment{read a word from the byte array}
\State $word \gets word \mathbin{\&} mask$ 


\end{algorithmic}
\caption{Pseudo code of DAC with select with block length 8} \label{select_pseudo}
\end{figure}

Figure~\ref{VBarrays} portrays an example of the bit array and the block array. $Select_1(k-1)$+1 returns the index of the starting byte of ith compressed element. Length of the compressed 
element is obtainable e.g. via $select_1(k)$. Then a word is read from the block array, starting from block $A_i$. Because element length in this case is 3, blocks not belonging to the element
are removed either with a bitmask or by double bit shifting, resulting in decoded k-th element.

\begin{figure}[ht]
\centering
\begin{tikzpicture}
[node distance=0pt,
 start chain = A going right,
 start chain = B going right,
 arrow/.style = {draw=#1,-{Stealth[]}, 
                shorten >=1mm, shorten <=1mm}, % styles of arrows
 arrow/.default = black,
 X/.style = {rectangle, draw,% styles of nodes in string (chain)
                minimum width=9ex, minimum height=5ex,
                outer sep=0pt, on chain}
 ]

\node[X, join] (start) {...};
\node[X] (first){0};
\node[X, very thick] (select){1};
\node[X] (Btarget){0};
\node[X, join] (second){0};
\node[X, very thick, join] (end){1};
\node[X, join] (third){0};
\node[X, join] {$B_{...}$};

\node [above=1ex of start]{Bit array};
\node [below=1ex of start]{index:};
\node [above=1ex of select]{k-1th 1};
\node [below=1ex of first]{$i-1$};
\node [below=1ex of select]{$i$};
\node [below=1ex of Btarget]{$i+1$};
\node [below=1ex of second]{$i+2$};
\node [below=1ex of end]{$i+3$};
\node [below=1ex of third]{$i+4$};
\node [above=1ex of end]{kth 1};
\node[X, below=12ex of start] (x) {...};
\node[X, join] {$A_{i-2}$};
\node[X, join] {$A_{i-1}$};
\node[X, ultra thick] (Atarget){$A_{i}$};
\node[X, ultra thick] {$A_{i+1}$};
\node[X, ultra thick] {$A_{i+2}$};
\node[X, join] {$A_{i+3}$};
\node[X, join] {$A_{...}$};
\node [above=1ex of x]{Block array};


\draw[arrow] (Btarget.south) to (Atarget.north);
\end{tikzpicture}
\caption{Data} \label{VBarrays}
\end{figure}

\section{Elias-Fano encoding}
Direct access via select can also be implemented with Elias-Fano (TODO citation to Elias-Fano?), but the data has to be sorted. In Elias-Fano encoding, the data is at first split into low and 
high bits. Low bits, the k least significant bits, are stored to their own array. Direct access to the lower bits is trivial. The higher bits are gap-encoded and then encoded in unary. 
For example, [1,3,5,8,11] becomes [1,2,2,3,3] and then is encoded as unary bit array B = \texttt{0100100100010001}. $Select_1$ has an interesting interaction with gap encoded unary arrays. $k = Select_1(B,i)$ 
returns the location of i-th 1 in B. Up to index $k$, the bit array has had $i$ ones and $i-k$ zeros. The $i$ corresponds to the number of compressed integers in $B$ and every zero in the array 
means an increment of 1 in the numbers. In other words, $i-k$ equals to the integer at position $i$ before gap-encoding. Therefore $select_1(i)$ - i returns the original higher bit number. The split 
between bits is made to reduce the gap sizes.
This approach can be further optimized by finding clusters in the integers and partitioning the indexes \citep{Ott14}.

\chapter{Sub-array access}
After locating and decompressing one value from a compressed data structure, compressing the subsequent items gets a lot easier. Because the next compressed value is usually located next to the fetched
in memory, locating the data is not needed. This is especially beneficial to the VByte decoding with $select$ method, where calculating the start byte location takes majority of he runtime. In 
addition to this, the length of the next compressed integer is faster to calculate because already manipulated data in memory. 

The shown $rank$ method by Brisaboa et al. can also effectively decompress the next value but some modifications are needed. When fetching a longer subarray, the goal becomes to minimize $rank$ calls. 
If an additional pointer array is stored during subarray decompression, only one $rank$ call for each accessed block level is needed. This can be observed from Figure~\ref{bris_ds}; the blocks of subarray
are located next to each other in different block levels.

 - examples / visualizaton?



 - TODO get results from this
\chapter{Experimental results}

The following experiments were run on a AMD Phenom(tm) II X6 1055T Processor@2.8Ghz with 64kB+64kB L1 cache, 512kB L2 cache, 6144kB L3 cache and 32GB of DDR3 1333MHz.
The computer runs Ubuntu 14.04.4 (3.13.0-91-generic x86\_64). The codes were compiled with \texttt{g++-7 -std=c++11 -DNDEBUG} and with libraries \texttt{-lsdsl -ldivsufsort -ldivsufsort64}.

Five different VB decoding algorithms were compared. The Bris-algorithms are from \citep{Bri09}. All algorithms are implemented using data structures and functions from the SDSL 
library \citep{gbmp2014sea}. Each implementation is compressing and decoding uint64\_t type integers.

\begin{itemize}
  \item Bris4v5 - Bris implementation with block size 4, using rank support v5
  \item Bris8 - Bris implementation with block size 8, using rank support v
  \item Bris8v5 - Bris implementation with block size 8, using rank support v5
  \item Select4 - Proposed implementation with block size 4, using select support mcl
  \item Select8 - Proposed implementation with block size 8, using select support mcl
\end{itemize}






Four differently constructed datasets were used and three different sizes of dataset were used (5M, 50M, 500M). 

\begin{itemize}
  \item all - all integers randomly 1-4 bytes long
  \item twolarge - one eighth of integers 4 bytes long, one eighth 2 bytes long, rest 1 byte    
  \item onelarge - one eighth of integers 2 bytes long, rest 4 bits or less
  \item onlysmall - all integers 4 bits or less
\end{itemize}

All implementations read the data set from a file and compress it to memory, randomize one million indexes and stored to an array. The times shown in Table~\ref{table:results1} 
are times taken from looping through the index array and VB decoding the number in the index. Additionally, each run is iterated multiple times and an average of the runtimes is taken. 

\begin{table}
\centering
\caption{Results in milliseconds, smaller is better.\label{table:results1}}
\begin{tabular}{l||c c c c c c c} 
& bris8 & bris4 & bris8v5 & bris4v5 & select8 & select4  \\
 \hline \hline 
all (5M)   & 191.80 & 471.50 & 234.00 & 569.40 & 149.20 & 175.90 \\
all (50M)   & 269.20 & 573.80 & 323.00 & 785.00 & 245.40 & 275.00 \\
all (500M)   & 298.50 & 677.80 & 368.20 & 941.80 & 355.90 & 382.20 \\
twolarge (5M)   & 96.70 & 275.50 & 110.70 & 329.00 & 125.60 & 151.90 \\
twolarge (50M)   & 140.40 & 379.70 & 159.80 & 456.80 & 227.70 & 242.20 \\
twolarge (500M)   & 168.20 & 441.00 & 191.50 & 584.70 & 330.20 & 341.00 \\
onelarge (5M)   & 25.80 & 40.10 & 28.30 & 49.20 & 95.90 & 100.40 \\
onelarge (50M)   & 49.80 & 91.30 & 50.90 & 105.70 & 212.70 & 213.20 \\
onelarge (500M)   & 55.00 & 100.00 & 58.50 & 124.00 & 315.80 & 307.70 \\
onlysmall (5M)   & 16.50 & 20.20 & 16.50 & 23.80 & 84.70 & 88.40 \\
onlysmall (50M)   & 31.20 & 51.80 & 32.10 & 57.00 & 201.20 & 198.80 \\
onlysmall (500M)   & 34.10 & 56.00 & 35.00 & 63.10 & 297.90 & 293.80 \\




\hline
%
\end{tabular}
\end{table}

The results show that the proposed $select$ method outperforms $rank$ when the data set contains several integers that are encoded into more than one block. The difference is explained 
with how $rank$ and $select$ use the data structure. $Select$ is always called once, while $rank$ is called once for each block beyond the first. The 8-bit versions tend to perform better than the 
4-bit versions. This is because 8-bit version stores more data in one block and less block readings are required. However, 4bit versions achieve better compression in data sets containing small integers.
Select versions do not depend as much on the block size, because all the blocks in 
the compressed element are read at one go. 

In rank, the v5 version naturally has worse performance because more calculations are needed to locate the correct index. As a trade-off, it requires 
significantly less memory, as seen on Figure~\ref{table:supportsize}.

Interestingly the 4-bit version of $select$ performs better than the 8-bit version with some data sets, even though 4-bit version should execute at least the same commands as the 8-bit version 
in every scenario. This is most likely due to cache, because in the 4-bit version the stored data set is smaller.


 - TODO figure out why 4bit crashes sometimes with 500M\_all

 - see how data length percentage matters (do datasets with k/8 long numbers).

 - see how cache matters

 - better dataset might be something with larger (like 60bit) numbers

 
 - See Ahn \& Moffat 2005; Lemire et al. 2018

 - subarray access(?)
 
 - bris has advantage of different size blocks (in different levels)

 - get compression sizes of 4 and 8 bit versions

 - figure out why data size matters

 - do we need to support search()?

 - $rank$ vs rank? should formulas have a specific style?

\chapter{Future work}

There has already been a few successful attempts to decompress VB encoded integers with SIMD \citep{Lem18,Pla15}. It is worthwhile to check if the presented approach benefits from SIMD. The data needs 
less manipulation because the data is already written as it is, but calculating the integer block lengths might benefit from SIMD.

Another improvement to VB encoding is to see whether the compressed numbers can be changed. This is trivial until the block length of the compressed number changes. Then an additional data structure or 
cache is needed and the supporting rank/select data structure needs to be recalculated.


\chapter{Conclusions\label{chapter:conclusions}}

It is good to conclude with a summary of findings. You can also use separate chapter for discussion and future work. These details you can negotiate with your supervisor.
